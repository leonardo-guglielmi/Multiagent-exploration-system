\section{Modello del sistema} \label{sec:modello_sistema}
Si considerino $M$ \textbf{utenti} a cui deve essere fornita connessione alla rete, distribuiti a terra all'interno di un'area bidimensionale $A$, con posizioni $u_j \in \mathbbm{R}^2 ,\ j \in \{1, ...M \}$.
Sono presenti $N$ \textbf{agenti} mobili UAV ed eventualmente $B$ \textbf{Base Stations}, aventi l'obiettivo di formare una rete mobile UAV e di fornire segnale agli utenti, mentre esplorano quelle porzioni di area che non sono coperte.
Si indicano con $x_i \in \mathbbm{R}^3 ,\ i \in \{1, ...N \}$ le posizioni degli agenti mobili, e con $b_k \in \mathbbm{R}^2 ,\ k \in \{1, ...B\}$ le posizioni delle BS.
Per limitare la problematica della connessione tra agenti, si ipotizza l'impiego di una rete di backhaul che permetta di considerare il grafo di connessione tra agenti sempre connesso.
D'ora in poi, per semplificare la trattazione, si userà il termine \textit{sensore} per indicare sia gli agenti mobili che le BS, in quanto entrambi hanno la capacità di fornire copertura ed esplorare.

Vengono quindi fatte le seguenti ipotesi sul sistema:
\begin{enumerate}

\item
il numero di agenti mobili è strettamente minore del numero di utenti, così come il numero di BS, se presenti, è minore di quello degli agenti: $B<N<M$

% ============================================================================================
% OLD GRAPH SENSOR NETWORK
% ============================================================================================
%\item 
%ciascun agente ha una capacità di comunicazione limitata, modellata mediante una sfera di raggio $R$ centrata nella posizione dell'agente; si ha che, affinché due sensori $i$ e $j$ possano comunicare direttamente, deve intercorrere una distanza $d(i,j)<R$

%\item
%la rete che si forma tra i sensori è modellata tramite un grafo indiretto $G=(V,E(t))$, dove $V$ rappresenta l'insieme dei nodi (agenti mobili e BS) mentre $E(t)$ rappresenta i collegamenti indiretti attivi tra i sensori; deriva dunque che, al tempo $t$, $(i,j)\in E(t)\iff d(i,j)<R$.

%\item
%la rete di sensori comunica mediante una topologia decentralizzata; ciò implica che sia sufficiente che il grafo sia connesso in ciascun istante $t$ 

\item
al fine di evitare collisioni, gli agenti sono disposti ad altitudini diverse, da cui si ha che, per un sensore $i$ e un utente $j$, $d_{i,j}(t)=\sqrt{||x_i - u_j ||^2 + h_i^2}$

\item
tra sensori e utenti vengono considerati i collegamenti LoS

\end{enumerate}