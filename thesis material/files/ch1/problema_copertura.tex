\subsection{Problema della copertura} \label{subsec:prob_copertura}

Si consideri un insieme di utenti, posizionati a terra, di cui non si conosce il numero complessivo né la posizione, a cui si deve fornire connessione tramite la rete di droni.
Il problema di copertura si pone l'obiettivo di trovare una disposizione delle fonti di segnale (droni ed eventuali BS) tale che il segnale ricevuto da ciascun utente noto soddisfi una serie di requisiti.

Il controllo della copertura può ricadere sotto due categorie: statico e dinamico. 
Nel primo, lo scopo è quello di trovare il  posizionamento ottimale degli agenti per garantire la copertura degli utenti, mentre il secondo usufruisce della mobilità delle fonti di segnale, cercando di connettere dei soggetti che possono muoversi nel tempo.
In questa tesi, l'algoritmo di copertura statico \cite{tesiInnocenti} utilizzato nel lavoro precedente è stato riadattato per renderlo compatibile con lo scenario in cui il numero totale di utenti non è noto.