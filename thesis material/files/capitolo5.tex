\chapter{Conclusioni} \label{ch:conclusioni}
Gli esperimenti condotti hanno mostrato come, sotto le ipotesi di non conoscere il numero totale di agenti e che tale numero possa variare, l'algoritmo di esplorazione migliora i risultati ottenuti in termini di utenti coperti rispetto al sistema con il solo algoritmo di copertura.
Inoltre si è verificata l'adattabilità del sistema nel caso in cui venga usato con ambienti molto mutevoli e, nonostante un effettivo calo dei risultati, permette di raggiungere dei risultati accettabili.


Il sistema esposto in questa tesi e le relative analisi coprono solo alcuni casi d'uso, e pertanto le estensioni che è possibile effettuare sono molteplici.
Si potrebbe, per esempio, andare a verificarne il comportamento nel caso in cui gli utenti possano muoversi, senza però che il sistema conosca la frequenza o la lunghezza di tali spostamenti, includendo all'interno dell'algoritmo di controllo un metodo che permetta la stima di tali grandezze mediante un sistema in retroazione.
Oppure si potrebbe rilassare l'ipotesi che l'altitudine sia sufficiente da trascurare eventuali ostacoli orizzontali, e quindi verificare il comportamento del sistema in ambienti più complessi, come quelli urbani, facendo eseguire ai droni spostamenti non più solo bidimensionali, ma anche lungo l'asse verticale.
Un altro aspetto non verificato da questa tesi è quello del consumo energetico, che tuttavia gioca un ruolo fondamentale nei sistemi reali e che potrebbe essere incluso nel modello bilanciando il guadagno che si ottiene a seguito di uno spostamento con il suo consumo energetico.
Infine, è possibile sperimentare nuovi metodi per l'esplorazione, usando anche tecniche di machine learning, fra cui il reinforcement learning.