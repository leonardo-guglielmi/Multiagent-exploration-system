\section*{Struttura della tesi} \label{sec:struttura_tesi}

Il contenuto di questa tesi è strutturato nel seguente modo: all'interno del \textbf{capitolo 1} vengono introdotti alcuni concetti fondamentali, legati agli argomenti inerenti al contesto della tesi, quali una breve definizione di sistema multiagente e dei tipi di problema che vengono affrontati. Il \textbf{capitolo 2} si concentra sul presentare il modello matematico usato per formalizzare il problema e il sistema; successivamente nel \textbf{capitolo 3} si mostra l'implementazione dell'algoritmo di controllo e le strategie risolutive adottate.
Il \textbf{capitolo 4} riporta lo svolgimento degli esperimenti e l'analisi dei risultati da essi ottenuti.
Infine, nel \textbf{capitolo 5} vengono tratte le conclusioni sull'esperimento, soffermandosi anche su possibili futuri sviluppi della ricerca sul tema.