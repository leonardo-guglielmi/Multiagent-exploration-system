\section{Esecuzione concorrente} \label{subsec:concur_agents}

In un contesto reale, ciascun agente eseguirebbe la funzione obiettivo sul proprio sistema di calcolo in modo indipendente dagli altri.
Si è dunque ritenuto opportuno implementare un meccanismo di parallelizzazione per il calcolo del \texttt{goal\_point} di ciascun agente, riducendo sensibilmente i tempi di esecuzione delle simulazioni.
Data la presenza nell'interprete Python del \textit{Global Interpreter Lock}, la scelta del modulo da impiegare è ricaduta su \texttt{multiprocessing}, un package Python che espone delle API per la creazione e gestione di processi paralleli.

 Per parallelizzare la simulazione, si crea un processo distinto per ciascun agente, il quale eseguirà il metodo \texttt{goal\_point\_agent()}.
 Una volta calcolato il punto ottimo, il processo lo inserisce in un dizionario condiviso, associandogli come chiave l'id del proprio agente.
Una volta terminati tutti i processi agente, nel processo principale viene estratto dal dizionario ciascun \texttt{goal\_point} e associato al relativo agente (Snippet \ref{snip:concurrent_simu}).
\lstinputlisting[
language=Python 
, label = {snip:concurrent_simu}
, caption = {Esecuzione parallela della ricerca del \texttt{goal\_point}.}
, float = t
, frame = tb
, belowcaptionskip = 3mm
]{code/concurrent_simu.py}