\subsection{Valutazione dell'esplorazione tramite frontiera con controllo delle celle adiacenti} \label{subsec:expl_eval_frontier_NCC}
Il metodo riportato in questa sotto-sezione è quello che verrà usato nelle simulazioni esposte nel Capitolo \ref{ch4:simulazioni}.
Esso è la combinazione delle varianti \textit{LCIE} e \textit{LSIENCC}, esposte nella sotto-sezione precedente; tale metodo include dunque i vantaggi di avere una regione da valutare il più ridotta possibile, senza però rinunciare alla precisione della stima, e di poter predire il livello di copertura di alcune celle, senza doverne calcolare esplicitamente il valore.

% da ricontrollare
I passaggi principali in cui l'algoritmo si articola sono i seguenti:
\begin{enumerate}[wide]


\item
Per prima cosa, il metodo seleziona quelle celle che rientrano nell'area da valutare, inserendo le informazioni rilevanti (posizione del centro e probabilità) in una struttura dati simile ad una matrice, ma con lunghezza delle righe variabile (Snippet \ref{snip:init_expl_eval}).
Inoltre, identifica gli agenti abbastanza vicini al punto campionato.
\lstinputlisting[
language=Python 
, label={snip:init_expl_eval}
, caption={Inizializzazione del metodo LCIENCC.}
, frame=tb
, float = ht
, belowcaptionskip=3mm
]{code/init_expl_eval_method.py}


\item
Successivamente, per ogni cella inclusa, calcola il SINR di ciascun agente precedentemente selezionato.
Dal calcolo del SINR vengono escluse quelle celle aventi $P_k=0$,  in quanto sono già coperte, e non apporterebbero nessun contributo all'esplorazione (Snippet \ref{snip:core_expl_eval}).
Se il livello di SINR di una cella supera una certa soglia \texttt{NEIGHBOUR\_SINR\_THRESHOLD}, la cella sovrastante e quella a destra (quella sotto e quella a sinistra sono state già esaminate) vengono inserite in \texttt{already\_checked\_cells}, ossia una lista che permette di escludere tali celle dalla valutazione del livello di SINR.
Prima di tale inserimento, viene fatta una serie di controlli per escludere quelle celle che eccedono l'aera di valutazione, e per evitare di etichettare come coperta una cella avente probabilità zero. 
\lstinputlisting[
language=Python 
, label={snip:core_expl_eval}
, caption={Calcolo della copertura nel metodo LCIENCC.}
, frame=tb
, float = p
, belowcaptionskip=3mm
]{code/core_expl_eval_method.py}


\item
Infine si calcola il livello di esplorazione, basandosi sulla matrice che associa a ciascuna cella il valore di SINR di ogni agente. (Snippet \ref{snip:final_expl_eval}).
\lstinputlisting[
language=Python 
, label={snip:final_expl_eval}
, caption={Valutazione dell'esplorazione nel metodo LCIENCC.}
, frame=tb
, float = h
, belowcaptionskip=3mm
]{code/final_expl_eval_method.py}


\end{enumerate}