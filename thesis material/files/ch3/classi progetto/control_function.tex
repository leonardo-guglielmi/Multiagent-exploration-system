\subsection{Classe \texttt{Control\_function}} \label{subsec:control_function}
Questa classe contiene i metodi utilizzati dalla funzione di controllo.
Al suo interno sono presenti i riferimenti a tutti gli attori del sistema: troviamo infatti una lista di \texttt{Agent}, una di \texttt{Base\_station} e una di \texttt{User}.
Inoltre, contiene tutta una serie di attributi usati durante i processi decisionali, come il criterio per la scelta tra i punti campionati, oppure la matrice di probabilità utilizzata per l'esplorazione, descritta meglio in  \ref{subsec:mappa_prob}.

\subsubsection{Metodo  \texttt{move\_agents()}}
Questa funzione, mostrata nello Snippet \ref{snip:move_agents}, esegue lo spostamento degli agenti come descritto in \ref{sec:algoritmo_controllo}.
Inoltre, nel caso in cui venga riconosciuta una situazione di accoppiamento tra agenti, viene applicata una deviazione; tale processo è descritto approfonditamente in \ref{sec:controllo_accoppiamento_agenti}.

\lstinputlisting[
    language=Python 
    , label= {snip:move_agents}
    , caption = {Funzione per lo spostamento degli agenti.}
    , frame=tb
    , belowcaptionskip=3mm
    , float = h
    ]
{code/move_agent_snip.py}

\subsubsection{Metodo \texttt{get\_points()}}
Dato un agente, questa funzione campiona un insieme di punti e ne restituisce un sottoinsieme secondo una certa regola.
Più precisamente, vengono campionati \texttt{NUM\_OF\_SAMPLES} punti secondo una \textbf{distribuzione normale}, centrata nell'attuale posizione dell'agente specificato.
Le strategie di selezione dei punti adottate sono le seguenti:
\begin{enumerate}

\item
Ricerca \textbf{systematic}: vengono considerati tutti i punti campionati. 
Questa strategia ha il problema che, occasionalmente, restituisce dei punti lontani dall'agente, e se tra questi vi fosse il punto ottimo, porterebbe l'agente a muoversi verso zone che, probabilmente, verrebbero esplorate prima da altri agenti, rendendo lo spostamento vano.

\item
Ricerca \textbf{local}: tra i punti campionati, vengono selezionati solo quelli che sono più vicini all'agente specificato rispetto ad altri sensori (Snippet \ref{snip:local_search}). Questa strategia predilige quindi un controllo locale dell'area, impedendo che un sensore vada a esplorare una zona che è più facilmente esplorabile da un altro.

\lstinputlisting[
    language=Python 
    , label= {snip:local_search}
    , caption = {Tecnica di selezione \textit{local}.}
    , frame=tb
    , belowcaptionskip=3mm
    , float = t
    ]
{code/local_search_snip.py}

\item
Ricerca \textbf{penalty}: come nella strategia \textit{local}, questa tecnica favorisce quei punti prossimi all'agente. A differenza della precedente non va ad escludere quelli più vicini ad altri, bensì va ad applicare una penalità ai livelli di $C(t)$ e $\Pi(t)$ calcolati in tali punti.

\pagebreak
\item
Ricerca \textbf{annealing forward}: riprendendo l'idea dell'algoritmo di ricerca locale \textit{simulated annealing}, si va ad aggiungere alla strategia \textit{local} una probabilità aggiuntiva, decrescente con il tempo, che il punto venga selezionato anche se è più vicino ad altri agenti. %(Snippet \ref{snip:annealing_search}).
Questa tecnica cerca di evitare lo stallo in punti di massimo locale.

\item
Ricerca \textbf{annealing reverse}: simile alla ricerca \textit{annealing forward}, in questa tecnica si va ad avere una probabilità crescente nel tempo. %(Snippet \ref{snip:annealing_search}).
In questo modo, con il passare del tempo, il valore atteso della lunghezza degli spostamenti aumenterà.
\end{enumerate}

%\lstinputlisting[
%    language=Python 
%    , label= {snip:annealing_search}
%    , caption = {Tecnica di selezione \textit{annealing forward} e \textit{annealing reverse}}
%    , frame=tb
%    , belowcaptionskip=3mm
%    , float = t
%    ]
%{code/annealing_search_snip.py}

\subsubsection{\texttt{find\_goal\_point\_for\_agent()}}
Dato un agente, questo metodo esamina tutti i punti forniti da \texttt{get\_points()}, più la posizione attuale dell'agente, cercando tra di essi l'ottimo.
Per fare ciò la posizione del sensore viene temporaneamente modificata con quella del punto in esame e viene ricalcolata la funzione obiettivo nella nuova configurazione del sistema; nel caso in cui il valore ottenuto sia il migliore trovato tale punto viene salvato come nuovo $p_i^*$.
Poiché la funzione obiettivo, come esposto nel capitolo \ref{ch:modello}, si divide in due parti, per valutare $R(t)$ nel nuovo punto dovranno essere calcolati nuovamente l'RCR e il valore di esplorazione.

\subsubsection{\texttt{is\_cell\_covered()}}
Questo metodo, fornite le coordinate di una cella di esplorazione, ritorna lo stato di copertura di quest'ultima.
Come indicato in \ref{sec:modello_esplorazione}, per determinare lo stato si prende come punto di riferimento il centro della cella; questo implica un'approssimazione sulla reale probabilità che in quella zona vi sia un utente, approssimazione che cresce all'aumentare della dimensione delle celle.

\input{files/ch3/classi progetto/mappa_probabilità}
\subsubsection{Funzione per l'esplorazione} \label{subsubsec:expl_funct}
Questa funzione (Snippet \ref{snip:expl_funct}), data la matrice di probabilità, va a calcolare il livello globale di esplorazione, andando a fare la media delle probabilità di tutte le celle e trasformando il risultato in modo che, ad una probabilità complessiva minore, venga associato un livello di esplorazione maggiore, rendendolo compatibile con un problema di massimizzazione (Formula \ref{eq:global_funct}).
\lstinputlisting[
language=Python 
, label={snip:expl_funct}
, caption={Funzione per il calcolo dell'esplorazione globale.}
, float = t
, frame=tb
, belowcaptionskip=3mm
]{code/global_expl_function.py}

Questo metodo può essere usato anche per valutare l'esplorazione in un punto campionato da un agente; tuttavia nella pratica quest'ultimo esplorerà solo quelle celle vicino al \texttt{goal\_point}, rendendo quindi il calcolo globale dell'esplorazione poco conveniente a causa del suo costo computazionale.
Per tale scopo sono stati quindi creati dei metodi di valutazione dell'esplorazione locale, descritti approfonditamente in \ref{sec:valutazione_esplorazione}.