\subsubsection{Matrice di distribuzione delle probabilità} \label{subsec:mappa_prob}
Questa matrice rappresenta l'area di interesse vista dal processo di esplorazione, come riportato in \ref{sec:modello_esplorazione}.
Essa associa a ciascuna cella, di forma quadrata di lati \texttt{EXPLORATION\_CELL\_WIDTH} e \texttt{EXPLORATION\_CELL\_HEIGTH}, la probabilità che al suo interno vi sia un utente non coperto.

Ad ogni iterazione la matrice di mappatura deve essere aggiornata in modo da riflettere le variazioni dello stato di copertura delle celle; a tale scopo è stata definita una procedura (Snippet \ref{snip:prob_matrix}) che esamina ogni cella, valuta il suo stato di esplorazione con il metodo \texttt{is\_cell\_covered()}, e in caso non sia coperta va ad aggiornarne la probabilità secondo la Formula \ref{eq:prob_cella}.
Inoltre tale funzione, nel caso in cui al seguito dello spostamento di un sensore un utente che si trova in una cella $k$ passasse dallo stato coperto a quello non coperto, aggiorna la probabilità di quella cella, impostando $P_k=1$; infatti al seguito del cambio di configurazione si ha la certezza che in quella zona vi sia un utente.
\lstinputlisting[
language=Python 
, label={snip:prob_matrix}
, caption={Metodo di aggiornamento della matrice di probabilità.}
, float = p
, frame=tb
, belowcaptionskip=3mm
]{code/prob_matrix_snip.py}